\documentclass[a4paper,12pt]{article}
\usepackage[T2A]{fontenc}
\usepackage[utf8]{inputenc}
\usepackage[english,russian]{babel}
% пакеты
\usepackage{amssymb,amsfonts,amsmath,mathtext,cite,enumerate,float,indentfirst}

\usepackage{xcolor}
\definecolor{linkcolor}{HTML}{000000}

\usepackage{hyperref}
\hypersetup{pdfstartview=FitH, linkcolor=linkcolor, colorlinks=true}

\usepackage[final]{graphicx} % пакет для рисунков
\graphicspath{{images/}} %путь к рисункам

\usepackage{geometry} % Меняем поля страницы
\geometry{left=2cm}% левое поле
\geometry{right=1.5cm}% правое поле
\geometry{top=1cm}% верхнее поле
\geometry{bottom=2cm}% нижнее поле

% Меняем везде перечисления на цифра.цифра
\renewcommand{\theenumi}{\arabic{enumi}}
\renewcommand{\labelenumi}{\arabic{enumi})}
\renewcommand{\theenumii}{.\arabic{enumii}}
\renewcommand{\labelenumii}{\arabic{enumi}.\arabic{enumii}.}
\renewcommand{\theenumiii}{.\arabic{enumiii}}
\renewcommand{\labelenumiii}{\arabic{enumi}.\arabic{enumii}.\arabic{enumiii}.}
\renewcommand{\thesection}{\arabic{section}.}
\renewcommand{\thesubsection}{\arabic{section}.\arabic{subsection}}
\renewcommand{\thesubsubsection}{\arabic{section}.\arabic{subsection}.\arabic{subsubsection}}

\newcommand*{\No}{\textnumero}

\usepackage{color} %% отображение цвета в коде
\usepackage{listings} %% пакет listings

\definecolor{gray}{rgb}{0.5,0.5,0.5}
\definecolor{mygreen}{rgb}{0,0.6,0}
\definecolor{mygray}{rgb}{0.5,0.5,0.5}
\definecolor{mymauve}{rgb}{0.58,0,0.82}

\usepackage{caption}
\DeclareCaptionFont{white}{\color{white}} %% текст заголовка белым
%% серая рамка вокруг заголовка кода.
\DeclareCaptionFormat{listing}{\colorbox{gray}{\parbox{\textwidth}{#1#2#3}}}
\captionsetup[lstlisting]{format=listing,labelfont=white,textfont=white}

\lstset{ %
	language=C++,                 % выбор языка для подсветки (здесь это С)
	basicstyle=\small\sffamily, % размер и начертание шрифта для подсветки кода
	numbers=left,               % где поставить нумерацию строк (слева\справа)
	numberstyle=\tiny,           % размер шрифта для номеров строк
	stepnumber=1,                   % размер шага между двумя номерами строк
	numbersep=5pt,                % как далеко отстоят номера строк от подсвечиваемого кода
	backgroundcolor=\color{white}, % цвет фона подсветки - используем \usepackage{color}
	showspaces=false,            % показывать или нет пробелы специальными отступами
	showstringspaces=false,      % показывать или нет пробелы в строках
	showtabs=false,             % показывать или нет табуляцию в строках
	frame=single,              % рисовать рамку вокруг кода
	tabsize=2,                 % размер табуляции по умолчанию равен 2 пробелам
	captionpos=t,              % позиция заголовка вверху [t] или внизу [b] 
	breaklines=true,           % автоматически переносить строки (да\нет)
	breakatwhitespace=false, % переносить строки только если есть пробел
	escapeinside={\%*}{*)}   % если нужно добавить комментарии в коде
}

\begin{document}
	\begin{titlepage}
\begin{center}
 \hfill \break
 \textit{
  \normalsize{Государственное образовательное учреждение высшего профессионального образования}}\\ 
 
 \textit{
  \normalsize  {\bf  «Московский государственный технический университет}\\ 
  \normalsize  {\bf имени Н. Э. Баумана»}\\
  \normalsize  {\bf (МГТУ им. Н.Э. Баумана)}\\
 }
 \noindent\rule{\textwidth}{2pt}
 \hfill \break
 \noindent
 \makebox[0pt][c]{ФАКУЛЬТЕТ}
 \makebox[\textwidth][c]{«Информатика и системы управления»}
 \\
 \noindent
 \makebox[0pt][c]{КАФЕДРА}
 \makebox[\textwidth][c]{«Программное обеспечение ЭВМ и информационные технологии»}
    \vfill
 \normalsize{\bf РАСЧЁТНО - ПОЯСНИТЕЛЬНАЯ\space\space ЗАПИСКА}\\ \vspace{5mm}
    \normalsize{\bf к курсовому проекту по курсу:}\\ \vspace{5mm}
    \large{<<Компьютерная графика>>}\\ \vspace{5mm}
    \normalsize{\bf на тему:}\\\vspace{5mm}
    \large{<<Программа визуализации сочленения объектов>>}\\
    \vfill
 \normalsize {
  \noindent
  \makebox[0pt][l]{Студент}%
  \makebox[\textwidth][c]{}%
        \makebox[0pt][r]{{$\underset{\text{(Подипсь, дата)}}{\underline{\hspace{6cm}}}$ \space Гулая М. Д.}}
        \noindent
  \makebox[0pt][l]{Руководитель курсового проекта}%
  \makebox[\textwidth][c]{}%
  \makebox[0pt][r]{{$\underset{\text{(Подипсь, дата)}}{\underline{\hspace{5.5cm}}}$ Рязанова Н. Ю.}}
 }\\
\vfill
\end{center}
\vfill
\begin{center} Москва, 2018\end{center}
\end{titlepage}
	\tableofcontents
\end{document}