\documentclass[a4paper,14pt]{article}
\usepackage[14pt]{extsizes}
\usepackage[T2A]{fontenc}
\usepackage[utf8]{inputenc}
\usepackage[english,russian]{babel}
% пакеты
\usepackage{amssymb,amsfonts,amsmath,mathtext,cite,enumerate,float,indentfirst}

\usepackage{xcolor}
\definecolor{linkcolor}{HTML}{000000}

\usepackage{hyperref}
\hypersetup{pdfstartview=FitH, linkcolor=linkcolor, colorlinks=true}

\usepackage[final]{graphicx} % пакет для рисунков
\graphicspath{{images/}} %путь к рисункам

\usepackage{geometry} % Меняем поля страницы
\geometry{left=2cm}% левое поле
\geometry{right=1.5cm}% правое поле
\geometry{top=1cm}% верхнее поле
\geometry{bottom=2cm}% нижнее поле

% Меняем везде перечисления на цифра.цифра
\renewcommand{\theenumi}{\arabic{enumi}}
\renewcommand{\labelenumi}{\arabic{enumi})}
\renewcommand{\theenumii}{.\arabic{enumii}}
\renewcommand{\labelenumii}{\arabic{enumi}.\arabic{enumii}.}
\renewcommand{\theenumiii}{.\arabic{enumiii}}
\renewcommand{\labelenumiii}{\arabic{enumi}.\arabic{enumii}.\arabic{enumiii}.}
\renewcommand{\thesection}{\arabic{section}.}
\renewcommand{\thesubsection}{\arabic{section}.\arabic{subsection}.}
\renewcommand{\thesubsubsection}{\arabic{section}.\arabic{subsection}.\arabic{subsubsection}.}

\renewcommand{\baselinestretch}{1.3}

\newcommand*{\No}{\textnumero}
\newcommand{\anonsection}[1]{\section*{#1}\addcontentsline{toc}{section}{#1}}

\usepackage{color} %% отображение цвета в коде
\usepackage{listings} %% пакет listings

\definecolor{gray}{rgb}{0.5,0.5,0.5}
\definecolor{mygreen}{rgb}{0,0.6,0}
\definecolor{mygray}{rgb}{0.5,0.5,0.5}
\definecolor{mymauve}{rgb}{0.58,0,0.82}

\usepackage{caption}
\DeclareCaptionFont{white}{\color{white}} %% текст заголовка белым
%% серая рамка вокруг заголовка кода.
\DeclareCaptionFormat{listing}{\colorbox{gray}{\parbox{\textwidth}{#1#2#3}}}
\captionsetup[lstlisting]{format=listing,labelfont=white,textfont=white}

\lstset{ %
	language=C++,                 % выбор языка для подсветки (здесь это С)
	basicstyle=\small\sffamily, % размер и начертание шрифта для подсветки кода
	numbers=left,               % где поставить нумерацию строк (слева\справа)
	numberstyle=\tiny,           % размер шрифта для номеров строк
	stepnumber=1,                   % размер шага между двумя номерами строк
	numbersep=5pt,                % как далеко отстоят номера строк от подсвечиваемого кода
	backgroundcolor=\color{white}, % цвет фона подсветки - используем \usepackage{color}
	showspaces=false,            % показывать или нет пробелы специальными отступами
	showstringspaces=false,      % показывать или нет пробелы в строках
	showtabs=false,             % показывать или нет табуляцию в строках
	frame=single,              % рисовать рамку вокруг кода
	tabsize=2,                 % размер табуляции по умолчанию равен 2 пробелам
	captionpos=t,              % позиция заголовка вверху [t] или внизу [b] 
	breaklines=true,           % автоматически переносить строки (да\нет)
	breakatwhitespace=false, % переносить строки только если есть пробел
	escapeinside={\%*}{*)}   % если нужно добавить комментарии в коде
}

\begin{document}
	\begin{titlepage}
\begin{center}
 \hfill \break
 \textit{
  \normalsize{Государственное образовательное учреждение высшего профессионального образования}}\\ 
 
 \textit{
  \normalsize  {\bf  «Московский государственный технический университет}\\ 
  \normalsize  {\bf имени Н. Э. Баумана»}\\
  \normalsize  {\bf (МГТУ им. Н.Э. Баумана)}\\
 }
 \noindent\rule{\textwidth}{2pt}
 \hfill \break
 \noindent
 \makebox[0pt][c]{ФАКУЛЬТЕТ}
 \makebox[\textwidth][c]{«Информатика и системы управления»}
 \\
 \noindent
 \makebox[0pt][c]{КАФЕДРА}
 \makebox[\textwidth][c]{«Программное обеспечение ЭВМ и информационные технологии»}
    \vfill
 \normalsize{\bf РАСЧЁТНО - ПОЯСНИТЕЛЬНАЯ\space\space ЗАПИСКА}\\ \vspace{5mm}
    \normalsize{\bf к курсовому проекту по курсу:}\\ \vspace{5mm}
    \large{<<Компьютерная графика>>}\\ \vspace{5mm}
    \normalsize{\bf на тему:}\\\vspace{5mm}
    \large{<<Программа визуализации сочленения объектов>>}\\
    \vfill
 \normalsize {
  \noindent
  \makebox[0pt][l]{Студент}%
  \makebox[\textwidth][c]{}%
        \makebox[0pt][r]{{$\underset{\text{(Подипсь, дата)}}{\underline{\hspace{6cm}}}$ \space Гулая М. Д.}}
        \noindent
  \makebox[0pt][l]{Руководитель курсового проекта}%
  \makebox[\textwidth][c]{}%
  \makebox[0pt][r]{{$\underset{\text{(Подипсь, дата)}}{\underline{\hspace{5.5cm}}}$ Рязанова Н. Ю.}}
 }\\
\vfill
\end{center}
\vfill
\begin{center} Москва, 2018\end{center}
\end{titlepage}
	\tableofcontents
	\newpage
\section*{Введение}\label{sec:intro}

В настоящее время компьютерная графика применяется во множестве различных областей информационных технологий. В связи с этим перед ней стоит множество задач, одной из которых является решение проблемы сочленения объектов.\\

Сочленение объектов используется при разработке компьютерных игр, обработке видео, изображений, моделировании различных объектов реальной жизни.\\

В данной работе приведен анализ алгоритмов сочленения объектов (в том числе риггинга), анализ возможных трудностей, с которыми можно столкнуться в процессе имплементирования этих алгоритмов, а так же пример реализации одного из них.
	\newpage
\section{Аналитическая часть}\label{sec:analytic}

В данном разделе рассмотрены алгоритмы решения поставленной задачи и их анализ.

\subsection{Обзор алгоритмов удаления невидимых линий}\label{subsec:analytic1.1}
 
Задача удаления невидимых линий и поверхностей является одной из наиболее сложных в машинной графике.\\

Алгоритмы удаления невидимых линий и поверхностей служат для определения линий ребер, поверхностей или объемов, которые видимы или невидимы для наблюдателя, находящегося в заданной точке пространства.\\

Необходимость удаления невидимых линий, ребер, поверхностей или объемов проиллюстрирована на рисунке ниже.

\begin{figure}[H]
\center
\includegraphics[width=\linewidth]{/analytic/pic1}
\caption{Необходимость удаления невидимых линий.}
\end{figure}

\subsubsection{Алгоритм Робертса}\label{subsec:analytic1.1.1}

Алгоритм Робертса представляет собой первое известное решение задачи об удалении невидимых линий.\\

Алгоритм прежде всего удаляет из каждого тела те ребра или грани, которые экранируются самим телом. Затем каждое из видимых ребер каждого тела сравнивается с каждым из оставшихся тел для определения того, какая его часть или части, если таковые есть, экранируются этими телами.\\

\textbf{Работа алгоритма Робертса проходит в два этапа:}

\begin{enumerate}
	\item определение нелицевых граней для каждого тела отдельно;
	\item определение и удаление невидимых ребер.
\end{enumerate}

\textbf{Определение нелицевых граней:}\\

Пусть $F$ — некоторая грань многогранника. Плоскость, несущая эту грань, разделяет пространство на два подпространства. Назовем  положительным то  из них, в которое смотрит внешняя нормаль к грани. Если точка наблюдения – в положительном подпространстве, то грань – лицевая, в противном случае – нелицевая. Если многогранник выпуклый, то удаление всех нелицевых граней полностью решает задачу визуализации с удалением невидимых граней.\\

Для определения, лежит ли точка в положительном подпространстве, используют проверку знака скалярного произведения $(l, n)$, где $l$ – вектор, направленный к наблюдателю, фактически определяет точку наблюдения; n – вектор внешней нормали грани. Если $(l, n) > 0$, т. е. угол между векторами острый, то грань является лицевой. Если $(l, n) < 0$, т. е. угол между векторами тупой, то грань является нелицевой.\\

\textbf{Определение и удаление невидимых ребер:}\\

После первого этапа удаления нелицевых отрезков необходимо выяснить, существуют ли такие отрезки, которые экранируются другими телами в картинке или в сцене. Для этого каждый оставшийся отрезок или ребро нужно сравнить с другими телами сцены или картинки.
Возможны следующие случаи:

\begin{enumerate}
	\item грань ребра не закрывает (ребро остается в списке ребер);
	\item грань полностью закрывает ребро (ребро удаляется из списка рассматриваемых ребер);
	\item грань частично закрывает ребро (ребро разбивается на несколько частей, видимыми из которых являются не более двух, само ребро удаляется из списка рассматриваемых ребер, но в список проверяемых ребер добавляются те его части, которые данной гранью не закрываются).
\end{enumerate}

\begin{figure}[H]
\center
\includegraphics[scale=1]{/analytic/pic2}
\caption{Результат работы алгоритма.}
\end{figure}

Математические методы, используемые в этом алгоритме, просты, мощны и точны, однако, стоит заметить, что, из-за того, что алгоритм сравнивает каждое из видимых ребер каждого тела с каждым из оставшихся тел, вычислительная трудоемкость алгоритма Робертса растет теоретически, как квадрат числа объектов, что, соответственно, снижает скорость работы алгоритма.

\subsubsection{Алгоритм Варнока}\label{subsec:analytic1.1.2}

Алгоритм Варнока является одним из примеров алгоритма, основанного на разбиении картинной плоскости на части, для каждой из которых исходная задача может быть решена достаточно просто.\\

Поскольку алгоритм Варнока нацелен на обработку картинки, он работает в пространстве изображения. В пространстве изображения рассматривается окно и решается вопрос о том, пусто ли оно, или его содержимое достаточно просто для визуализации. Если это не так, то окно разбивается на фрагменты до тех пор, пока содержимое фрагмента не станет достаточно простым для визуализации или его размер не достигнет требуемого предела разрешения.\\

\textbf{Краткое описание оригинальной версии алгоритма, предложенного Варноком:}\\

Разобьем видимую часть картинной плоскости на четыре равные части и рассмотрим, каким образом могут соотноситься между собой проекции граней получившейся части картинной плоскости.\\

В итоге, возможны четыре различных случая, представленных на рисунке ниже.

\begin{figure}[H]
\center
\includegraphics[width=\linewidth]{/analytic/pic3}
\caption{Соотношение проекции грани с подокном.}
\end{figure}

\begin{enumerate}
	\item проекция грани полностью накрывает область (рис. 3, d);
	\item проекция грани пересекает область, но не содержится в ней полностью (рис. 3, с);
	\item проекция грани целиком содержится внутри области (рис. 3, b);
	\item проекция грани не имеет общих внутренних точек с рассматриваемой областью (рис. 3, a).
\end{enumerate}

Сравнивая область с проекциями всех граней, можно выделить случаи, когда изображение, получающееся в рассматриваемой области, определяется сразу:

\begin{enumerate}
	\item проекция ни одной грани не попадает в область;
	\item проекция только одной грани содержится в области или пересекает область (в этом случае проекции грани разбивают всю область на две части, одна из которых соответствует этой проекции);
	\item существует грань, проекция которой полностью накрывает данную область, и эта грань расположена к картинной плоскости ближе, чем все остальные грани, проекции которых пересекают данную область (в данном случае область соответствует этой грани).
\end{enumerate}

Если ни один из рассмотренных трех случаев не имеет места, то снова разбиваем область на четыре равные части и проверяем выполнение этих условий для каждой из частей. Те части, для которых таким образом не удалось установить видимость, разбиваем снова и т. д.\\

При помощи изложенного алгоритма можно удалить либо невидимые линии, либо невидимые поверхности. Однако простота критерия разбиения, а также негибкость способа разбиения приводят к тому, что количество подразбиений оказывается велико.

\subsubsection{Алгоритм, использующий Z-буфер}\label{subsec:analytic1.1.3}

Алгоритм, использующий Z-буфер, является одним из простейших с точки реализации алгоритмов удаления невидимых линий. Он работает в пространстве изображений. В алгоритме используются буфер кадра и буфер глубины.\\

В процессе работы глубина (значение координаты $z$) каждого нового пикселя, который надо занести в буфер кадра, сравнивается с глубиной того пикселя, который уже занесен в Z-буфер. Если это сравнение показывает, что новый пиксель расположен ближе к наблюдателю, чем пиксель, уже находящийся в буфере кадра, то новое значение координаты $z$ заносится в Z-буфер, корректируется значение интенсивности в буфере кадра.\\

Если известно уравнение плоскости, несущей каждый многоугольник, то вычисление глубины каждого пиксела на сканирующей строке можно проделать пошаговым способом. Грань при этом рисуется последовательно (строка за строкой). Для нахождения необходимых значений используется линейная интерполяция (рисунок ниже).

\begin{figure}[H]
\center
\includegraphics[scale=0.8]{/analytic/pic4}
\caption{Сканирующая строка по грани.}
\end{figure}

Для рисунка $y$ меняется от $y_{1}$ до $y_{2}$ и далее до $y_{3}$, при этом для каждой строки определяется $x_{a}$, $z_{a}$, $x_{b}$, $z_{b}$:

\begin{equation}
	x_{a} = x_{1} + (x_{2} - x_{1}) \cdot \frac{y - y_{1}}{y_{2} - y_{1}}; \notag
\end{equation}

\begin{equation}
	x_{b} = x_{1} + (x_{3} - x_{1}) \cdot \frac{y - y_{1}}{y_{3} - y_{1}}; \notag
\end{equation}

\begin{equation}
	z_{a} = z_{1} + (z_{2} - z_{1}) \cdot \frac{y - y_{1}}{y_{2} - y_{1}}; \notag
\end{equation}

\begin{equation}
	z_{b} = z_{1} + (z_{3} - z_{1}) \cdot \frac{y - y_{1}}{y_{3} - y_{1}}; \notag
\end{equation}

На сканирующей строке $x$ меняется от $x_{a}$ до $x_{b}$ и для каждой точки строки определяется глубина $z$:

\begin{equation}
	z = z_{a} + (z_{b} - z_{a}) \cdot \frac{x - x_{a}}{x_{b} - x_{a}}; \notag
\end{equation}

Реализация алгоритма вдоль сканирующей строки позволяет совместить алгоритм z-буфера с алгоритмами растровой развертки ребер и алгоритмами закраски грани.\\

На рисунке ниже продемонстрирована работа алгоритма.

\begin{figure}[H]
\center
\includegraphics[scale=0.8]{/analytic/pic5}
\caption{Протыкающий треугольник.}
\end{figure}

В начале в буфере кадра и в Z-буфере содержатся нули. После растровой развертки содержимое буфера кадра будет меняться.\\

Этот алгоритм делает тривиальной визуализацию пересечений поверхностей. Кроме того, объекты сцены могут рассматриваться в произвольном порядке, что позволяет не тратить время на сортировку объектов по глубине. Как уже и говорилось выше - реализация алгоритма вдоль сканирующей строки позволяет совместить его с алгоритмами закраски грани. Недостатком является большой объем требуемой памяти для хранения буферов.

\subsection{Обзор алгоритмов закраски}\label{subsec:analytic1.2}

В данном подразделе представлены существующие алгоритмы закраски объектов, которые обеспечивают реалистичное отображение объектов на экране.

\subsubsection{Алгоритм закраски Гуро}\label{subsec:analytic1.2.1}

Метод закраски, который основан на интерполяции интенсивности и известен как метод Гуро (по имени его разработчика), позволяет устранить дискретность изменения интенсивности.\\

Процесс закраски по методу Гуро осуществляется в четыре этапа:

\begin{enumerate}
	\item вычисление нормалей ко всем полигонам;
	\item определение нормалей в вершинах путем усреднения нормалей по всем полигональным граням, которым принадлежит вершина;
	\item используя нормали в вершинах и применяя произвольный метод закраски, вычисляются значения интенсивности в вершинах;
	\item каждый многоугольник закрашивается путем линейной интерполяции значений интенсивностей в вершинах сначала вдоль каждого ребра, а затем и между ребрами вдоль каждой сканирующей строки.
\end{enumerate}

\begin{figure}[H]
\center
\includegraphics[scale=0.8]{/analytic/pic6}
\caption{Усреднение нормалей к вершинам.}
\end{figure}

\begin{equation}
	N_{v} =  \frac{N_{1} + N_{2} + N_{3} + N{4} + N_{v}}{4} \notag
\end{equation}

Интерполяция вдоль ребер легко объединяется с алгоритмом удаления скрытых поверхностей, построенным на принципе построчного сканирования. Для всех ребер запоминается начальная интенсивность, а также изменение интенсивности при каждом единичном шаге по координате y, Заполнение видимого интервала на сканирующей строке производится путем интерполяции между значениями интенсивности на двух ребрах, ограничивающих интервал.

\begin{figure}[H]
\center
\includegraphics[scale=0.8]{/analytic/pic7}
\caption{Интерполяция интенсивностей.}
\end{figure}

\begin{equation}
    I_{a} = I_{1}\frac{y_{3} - y_{2}}{y_{1} - y_{2}} + I_{2}\frac{y_{1} - y_{3}}{y_{1} - y_{2}} \notag 
\end{equation}

\begin{equation}
    I_{b} = I_{1}\frac{y_{2} - y_{3}}{y_{1} - y_{3}} + I_{3}\frac{y_{2} - y_{3}}{y_{1} - y_{3}} \notag
\end{equation}

\begin{equation}
    I_{p} = I_{a}\frac{x_{b} - x_{p}}{x_{a} - x_{b}} + I_{b}\frac{x_{p} - x_{a}}{x_{b} - x_{a}} \notag
\end{equation}

\subsubsection{Алгоритм простой однотонной закраски}\label{subsec:analytic1.2.2}

При однотонной закраске вычисляют один уровень интенсивности, который используется для закраски всего многоугольника. При этом предполагается, что:

\begin{enumerate}
	\item источник света расположен в бесконечности, поэтому произведение векторов $(L * N)$ постоянно на всей полигональной грани;
	\item наблюдатель находится в бесконечности, поэтому произведение векторов $(N * V)$ постоянно на всей полигональной грани;
	\item многоугольник представляет реальную моделируемую поверхность, а не является аппроксимацией криволинейной поверхности (если какое-либо из первых двух предположений оказывается неприемлемым, можно воспользоваться усредненными значениями векторов $L$ и $V$, вычисленными, например, в центре многоугольника).
\end{enumerate}

Последнее предположение в большинстве случаев не выполняется, но оказывает существенно большее влияние на получаемое изображение, чем два других. Влияние состоит в том, что каждая из видимых полигональных граней аппроксимированной поверхности хорошо отличима от других, поскольку интенсивность каждой из этих граней отличается от интенсивности соседних граней. Различие в окраске соседних граней хорошо заметно вследствие эффекта полос Маха, что является значительным минусом данного алгоритма.

\subsubsection{Алгоритм Фонга}

В методе закраски, разработанном Фонгом, используется интерполяция вектора нормали к поверхности вдоль видимого интервала на сканирующей строке внутри многоугольника, а не интерполяция интенсивности как в методе Гуро, что позволяет снизить количество полос Маха.\\

Интерполяция выполняется между начальной и конечной нормалями, которые сами тоже являются результатами интерполяции вдоль ребер многоугольника между нормалями в вершинах. Нормали в вершинах, в свою очередь, вычисляются так же, как в методе закраски, построенном на основе интерполяции интенсивности.

\begin{figure}[H]
\center
\includegraphics[scale=0.8]{/analytic/pic8}
\caption{Сканирующая строка.}
\end{figure}

В каждом пикселе вдоль сканирующей строки новое значение интенсивности вычисляется с помощью любой модели закраски. Заметные улучшения по сравнению с интерполяцией интенсивности наблюдаются в случае использования модели с учетом зеркального отражения, так как при этом более точно воспроизводятся световые блики. Однако даже если зеркальное отражение не используется, интерполяция векторов нормали приводит к более качественным результатам, чем интерполяция интенсивности, поскольку аппроксимация нормали в этом случае осуществляется в каждой точке. При этом значительно возрастают вычислительные затраты.\\

Чтобы закрасить куски бикубической поверхности, для каждого пиксела, исходя из уравнений поверхности, вычисляется нормаль к поверхности. Этот процесс тоже достаточно дорогой. Затем с помощью любой модели закраски определяется значение интенсивности. Однако прежде чем применить метод закраски к плоским или бикубическим поверхностям, необходимо иметь информацию о том, какие источники света (если они имеются) в действительности освещают точку. Поэтому мы должны рассматривать также и тени.

\subsection{Обзор алгоритмов сочленения объектов}\label{subsec:analytic1.3}

В данном разделе подробно рассмотрены алгоритмы сочленения трехмерных объектов.

\subsubsection{Алгоритм смешивания вершин (Vertex Blending)}\label{subsec:analytic1.3.1}

Хотя этот подход подходит для моделей, подобных роботам, методы интерполяции могут использоваться для получения лучшего приближения поверхностей составления между движущимися частями. Процесс создания таких промежуточных поверхностей называется смешиванием вершин.\\

Соответствующие пары точек на двух движущихся частях могут быть соединены вместе, чтобы сформировать треугольный или четырехугольный элемент, принадлежащий промежуточной поверхности. Эти элементы могут быть дополнительно подразделены с использованием простой формулы линейной интерполяции, чтобы получить поверхность. Ниже представлены методы интерполяции более высокого порядка для генерации поверхностей смешения.

\begin{figure}[H]
\center
\includegraphics[scale=0.8]{/analytic/pic9}
\caption{Линейная интерполяция / Гермитова интерполяция.}
\end{figure}

Мы можем использовать кубические многочлены Безье для генерации интерполяционных кривых между движущимися частями с тангенциальной непрерывностью в конечных точках. Обозначают пару соответствующих точек на двух движущихся частях модели персонажа. $Q_{0}$ и $Q_{3}$ являются двумя точками на поверхностях, которые выбраны для определения локальных касательных направлений $P_{0}$ $Q_{0}$ и $P_{3}$ $Q_{3}$ соответственно.\\

Используя эти касательные направления, мы можем указать еще две точки, $P_{1}$ и $P_{2}$:

\begin{equation}
    P_{1} = P_{0} + \alpha \cdot (P_{0} - Q_{0}); \notag
\end{equation}

\begin{equation}
    P_{2} = P_{3} + \alpha \cdot (P_{3} - Q_{3}); \notag
\end{equation}

где $\alpha$ - положительная величина, используемая для увеличения или уменьшения длины касательных векторов $P_{1}-P_{0}$ и $P_{3}-P_{3}$. Точки на интерполяционной кривой Безье генерируются

\begin{figure}[H]
\center
\includegraphics[scale=0.8]{/analytic/pic10}
\caption{Интерполяция Безье / Гермитова интерполяция.}
\end{figure}

с использованием параметрического уравнения:

\begin{equation}
    Q_{t} = (1 - t)^3 \cdot P_{0} + 3 \cdot (1 - t)^2 \cdot t \cdot P_{1} + 3 \cdot (1 - t) \cdot t^2 P_{2} + t^2 \cdot P_{3}; \notag 
\end{equation}

Подставляя выражения:

\begin{equation}
    Q_{t} = (1 - 3 \cdot t^2 + 2 \cdot t^3) \cdot P_{0} + 3 \cdot (1 - t)^2 \cdot t \cdot \alpha (P_{0} - Q_{0}) + 3 \cdot (1 - t)^2 \cdot \alpha \cdot (P_{3} - Q_{3}) + (3 \cdot t^2 - 2 \cdot t^3) \cdot P_{3}; \notag
\end{equation}

Когда $\alpha$ увеличивается, вес касательных векторов на интерполирующей кривой увеличивается, и кривая становится ближе к касательным в конечных точках $P_{0}$, $P_{3}$. Следует позаботиться о том, чтобы точки $P_{0}$, $P_{1}$ находились на одной и той же стороне касательной $P_{2}-P_{3}$, и аналогичным образом точки $P_{2}$, $P_{3}$ лежали на одной и той же стороне касательной $P_{1}-P_{0}$. Установка большого значения $\alpha$ нарушает это условие, что приводит к искаженной кривой Безье.\\

Вторым методом интерполяции, который подходит для смешивания вершин, является интерполяция Гермита. Здесь касательные направления определяются с помощью векторов $P_{0}-Q_{0}$ и $Q_{3}-P_{3}$, а интерполяционная кривая задается как:

\begin{equation}
    H(t) = (1 - 3t^2 + 2t^3) \cdot P_{0} + (t - 2t^2 + t^3)\alpha(P_{0} - Q_{0}) \\
     + (-t^2 + t^3)\alpha(Q_{3} - P_{3}) + (3t^2 - 2t^3)P_{3}; \notag 
\end{equation}

Коэффициенты $P_{0}$, $P_{3}$ точно такие же, как и при интерполяции Безье. Поскольку касательные определены вдоль направления кривой от $P_{0}$ до $P_{3}$, интерполяция Гермита не имеет проблем, связанных с большими значениями.

\subsection{Вывод}\label{subsec:analytic1.4}

В данном разделе были рассмотрены и проанализированы алгоритмы, с помощью которых можно максимально эффективно и быстро решить поставленное техническое задание.\\

В ходе анализа алгоритмов было принято решение использовать алгоритм Z-буфера для удаления невидимых линий. Несмотря на то, что хранение данных буфера требует дополнительной памяти, в реалиях данного проекта это не является весомой причиной отказываться от данного алгоритма, ведь скорость и качество его работы превалирует над остальными методами удаления невидимых линий.\\

Для закраски объектов на сцене будет использовать метод Гуро. Из трех перечисленных методов, он является самым оптимальным вариантом в силу того, что скорость его работы быстрее, нежели скорость работы алгоритма метода Фонга. Это обеспечивается меньшим количеством вычислений, что приведет к увеличению быстродействия программы. Если говорить о простом алгоритме закраски полигонов - здесь Гуро тоже выигрывает, ведь благодаря вычислению интенсивности в точках он дает большую реалистичность изображения.\\
\end{document}