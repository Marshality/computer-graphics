\newpage
\section{Аналитическая часть}\label{sec:analytic}

В данном разделе рассмотрены алгоритмы решения поставленной задачи и их анализ.

\subsection{Обзор существующих алгоритмов удаления невидимых линий}\label{subsec:analytic1.1}
 
Задача удаления невидимых линий и поверхностей является одной из наиболее сложных в машинной графике.\\

Алгоритмы удаления невидимых линий и поверхностей служат для определения линий ребер, поверхностей или объемов, которые видимы или невидимы для наблюдателя, находящегося в заданной точке пространства.\\

Необходимость удаления невидимых линий, ребер, поверхностей или объемов проиллюстрирована на рисунке ниже.

\begin{figure}[H]
\center
\includegraphics[width=\linewidth]{/analytic/pic1}
\caption{Необходимость удаления невидимых линий.}
\end{figure}

\subsubsection{Алгоритм Робертса}\label{subsec:analytic1.1.1}

Алгоритм Робертса представляет собой первое известное решение задачи об удалении невидимых линий.\\

Алгоритм прежде всего удаляет из каждого тела те ребра или грани, которые экранируются самим телом. Затем каждое из видимых ребер каждого тела сравнивается с каждым из оставшихся тел для определения того, какая его часть или части, если таковые есть, экранируются этими телами.\\

\textbf{Работа алгоритма Робертса проходит в два этапа:}

\begin{enumerate}
	\item определение нелицевых граней для каждого тела отдельно;
	\item определение и удаление невидимых ребер.
\end{enumerate}

\textbf{Определение нелицевых граней:}\\

Пусть $F$ — некоторая грань многогранника. Плоскость, несущая эту грань, разделяет пространство на два подпространства. Назовем  положительным то  из них, в которое смотрит внешняя нормаль к грани. Если точка наблюдения – в положительном подпространстве, то грань – лицевая, в противном случае – нелицевая. Если многогранник выпуклый, то удаление всех нелицевых граней полностью решает задачу визуализации с удалением невидимых граней.\\

Для определения, лежит ли точка в положительном подпространстве, используют проверку знака скалярного произведения $(l, n)$, где $l$ – вектор, направленный к наблюдателю, фактически определяет точку наблюдения; n – вектор внешней нормали грани. Если $(l, n) > 0$, т. е. угол между векторами острый, то грань является лицевой. Если $(l, n) < 0$, т. е. угол между векторами тупой, то грань является нелицевой.\\

\textbf{Определение и удаление невидимых ребер:}\\

После первого этапа удаления нелицевых отрезков необходимо выяснить, существуют ли такие отрезки, которые экранируются другими телами в картинке или в сцене. Для этого каждый оставшийся отрезок или ребро нужно сравнить с другими телами сцены или картинки.
Возможны следующие случаи:

\begin{enumerate}
	\item грань ребра не закрывает (ребро остается в списке ребер);
	\item грань полностью закрывает ребро (ребро удаляется из списка рассматриваемых ребер);
	\item грань частично закрывает ребро (ребро разбивается на несколько частей, видимыми из которых являются не более двух, само ребро удаляется из списка рассматриваемых ребер, но в список проверяемых ребер добавляются те его части, которые данной гранью не закрываются).
\end{enumerate}

\begin{figure}[H]
\center
\includegraphics[scale=1]{/analytic/pic2}
\caption{Результат работы алгоритма.}
\end{figure}

Математические методы, используемые в этом алгоритме, просты, мощны и точны, однако, стоит заметить, что, из-за того, что алгоритм сравнивает каждое из видимых ребер каждого тела с каждым из оставшихся тел, вычислительная трудоемкость алгоритма Робертса растет теоретически, как квадрат числа объектов, что, соответственно, снижает скорость работы алгоритма.

\subsubsection{Алгоритм Варнока}\label{subsec:analytic1.1.2}

Алгоритм Варнока является одним из примеров алгоритма, основанного на разбиении картинной плоскости на части, для каждой из которых исходная задача может быть решена достаточно просто.\\

Поскольку алгоритм Варнока нацелен на обработку картинки, он работает в пространстве изображения. В пространстве изображения рассматривается окно и решается вопрос о том, пусто ли оно, или его содержимое достаточно просто для визуализации. Если это не так, то окно разбивается на фрагменты до тех пор, пока содержимое фрагмента не станет достаточно простым для визуализации или его размер не достигнет требуемого предела разрешения.\\

\textbf{Краткое описание оригинальной версии алгоритма, предложенного Варноком:}\\

Разобьем видимую часть картинной плоскости на четыре равные части и рассмотрим, каким образом могут соотноситься между собой проекции граней получившейся части картинной плоскости.\\

В итоге, возможны четыре различных случая, представленных на рисунке ниже.

\begin{figure}[H]
\center
\includegraphics[width=\linewidth]{/analytic/pic3}
\caption{Соотношение проекции грани с подокном.}
\end{figure}

\begin{enumerate}
	\item проекция грани полностью накрывает область (рис. 3, d);
	\item проекция грани пересекает область, но не содержится в ней полностью (рис. 3, с);
	\item проекция грани целиком содержится внутри области (рис. 3, b);
	\item проекция грани не имеет общих внутренних точек с рассматриваемой областью (рис. 3, a).
\end{enumerate}

Сравнивая область с проекциями всех граней, можно выделить случаи, когда изображение, получающееся в рассматриваемой области, определяется сразу:

\begin{enumerate}
	\item проекция ни одной грани не попадает в область;
	\item проекция только одной грани содержится в области или пересекает область (в этом случае проекции грани разбивают всю область на две части, одна из которых соответствует этой проекции);
	\item существует грань, проекция которой полностью накрывает данную область, и эта грань расположена к картинной плоскости ближе, чем все остальные грани, проекции которых пересекают данную область (в данном случае область соответствует этой грани).
\end{enumerate}

Если ни один из рассмотренных трех случаев не имеет места, то снова разбиваем область на четыре равные части и проверяем выполнение этих условий для каждой из частей. Те части, для которых таким образом не удалось установить видимость, разбиваем снова и т. д.\\

При помощи изложенного алгоритма можно удалить либо невидимые линии, либо невидимые поверхности. Однако простота критерия разбиения, а также негибкость способа разбиения приводят к тому, что количество подразбиений оказывается велико.

\subsubsection{Алгоритм, использующий Z-буфер}\label{subsec:analytic1.1.3}

Алгоритм, использующий Z-буфер, является одним из простейших с точки реализации алгоритмов удаления невидимых линий. Он работает в пространстве изображений. В алгоритме используются буфер кадра и буфер глубины.\\

В процессе работы глубина (значение координаты $z$) каждого нового пикселя, который надо занести в буфер кадра, сравнивается с глубиной того пикселя, который уже занесен в Z-буфер. Если это сравнение показывает, что новый пиксель расположен ближе к наблюдателю, чем пиксель, уже находящийся в буфере кадра, то новое значение координаты $z$ заносится в Z-буфер, корректируется значение интенсивности в буфере кадра.\\

Если известно уравнение плоскости, несущей каждый многоугольник, то вычисление глубины каждого пиксела на сканирующей строке можно проделать пошаговым способом. Грань при этом рисуется последовательно (строка за строкой). Для нахождения необходимых значений используется линейная интерполяция (рисунок ниже).

\begin{figure}[H]
\center
\includegraphics[scale=0.8]{/analytic/pic4}
\caption{Сканирующая строка по грани.}
\end{figure}

Для рисунка $y$ меняется от $y_{1}$ до $y_{2}$ и далее до $y_{3}$, при этом для каждой строки определяется $x_{a}$, $z_{a}$, $x_{b}$, $z_{b}$:

\begin{equation}
	x_{a} = x_{1} + (x_{2} - x_{1}) \cdot \frac{y - y_{1}}{y_{2} - y_{1}};
\end{equation}

\begin{equation}
	x_{b} = x_{1} + (x_{3} - x_{1}) \cdot \frac{y - y_{1}}{y_{3} - y_{1}};
\end{equation}

\begin{equation}
	z_{a} = z_{1} + (z_{2} - z_{1}) \cdot \frac{y - y_{1}}{y_{2} - y_{1}};
\end{equation}

\begin{equation}
	z_{b} = z_{1} + (z_{3} - z_{1}) \cdot \frac{y - y_{1}}{y_{3} - y_{1}};
\end{equation}

На сканирующей строке $x$ меняется от $x_{a}$ до $x_{b}$ и для каждой точки строки определяется глубина $z$:

\begin{equation}
	z = z_{a} + (z_{b} - z_{a}) \cdot \frac{x - x_{a}}{x_{b} - x_{a}};
\end{equation}

Реализация алгоритма вдоль сканирующей строки позволяет совместить алгоритм z-буфера с алгоритмами растровой развертки ребер и алгоритмами закраски грани.\\

На рисунке ниже продемонстрирована работа алгоритма.

\begin{figure}[H]
\center
\includegraphics[scale=0.8]{/analytic/pic5}
\caption{Протыкающий треугольник.}
\end{figure}

В начале в буфере кадра и в Z-буфере содержатся нули. После растровой развертки содержимое буфера кадра будет меняться.\\

Этот алгоритм делает тривиальной визуализацию пересечений поверхностей. Кроме того, объекты сцены могут рассматриваться в произвольном порядке, что позволяет не тратить время на сортировку объектов по глубине. Как уже и говорилось выше - реализация алгоритма вдоль сканирующей строки позволяет совместить его с алгоритмами закраски грани. Недостатком является большой объем требуемой памяти для хранения буферов.

